\documentclass[a4paper]{scrreprt}
\usepackage{fullpage} % Slightly more margins
\usepackage{amsfonts}
\usepackage{fancyhdr}
\pagestyle{fancy}
\usepackage[english]{babel}
\selectlanguage{english}
\usepackage[utf8]{inputenc}
\usepackage{graphicx}
\usepackage{url}
\usepackage{textcomp}
\usepackage{amsmath}
\usepackage{lastpage}
\usepackage{pgf}
\usepackage{wrapfig}
\usepackage{fancyvrb}
\usepackage{listings} % Source code highlighting
\usepackage{algpseudocode} % Algorithms
\usepackage{courier}

% Create header and footer
\headheight 27pt
\pagestyle{fancyplain}
\lhead{\footnotesize{Object-Oriented Design, IV1350}}
\chead{}
\rhead{\footnotesize{Assignment 3 Report}}
\lfoot{}
\cfoot{\thepage (\pageref{LastPage})}

% Create title page
\title{Assignment 3}
\subtitle{Object-Oriented Design, IV1350}
\author{Emil Tullstedt emiltu@kth.se}
\date{2014-04-29}

\graphicspath{ {./images/} }
\lstset{basicstyle=\footnotesize\ttfamily, language=Java, numbers=left}

\begin{document}

\maketitle

\tableofcontents %Generates the TOC

\chapter{Introduction}
This report describes the implementation of a previously designed cache simulator made for a pregraduate project in the course \textbf{IV1350} at KTH in Stockholm. The implementation is not for production usage.

\begin{small}
While implementing the application, I co-operated with \textit{Martin Alge}, \textit{Jesper Falk} and \textit{Erik Pettersson}.
\end{small}

\chapter{Method}
In the process of procuring a working application that would run an implementation of the design bescribed in \textit{Assignment 2}\footnote{http://web.ict.kth.se/$\sim$emiltu/iv1350-emiltu-s2.pdf} Java were selected as being the preferred language in the documentation of the course.

The Java environment during the course of developing the application have consisted of the following:

\begin{itemize}
	\item OpenJDK Runtime Environment (\textbf{IcedTea} 2.4.4) (7u51-2.4.4-0ubuntu0.13.10.1)
	\item \textbf{rake} 10.3.0
	\item \textbf{rakejava} 1.3.7
	\item \textbf{javac} 1.7.0\_51
	\item \textbf{git} 1.8.3.2
\end{itemize}

As the application was already largely designed, implementation was made simplified. As the design was having a largely low coupling - it was possible to implement the separate classes for the design one by one, thus rapidly having a working application. The application were mostly written after the design with modifications were seemingly proper. During a period in the development process, the classes \texttt{CacheLayout} and \texttt{AddressLayout} was merged in the \texttt{CacheLayout}-class, but that idea was quickly scrapped since the class were dangerously close to becoming a megamodel containing most of the logic in the application.

The tests made their way into the application quite late, which have caused a couple of hours of unnecessary debugging for bugs that would've been quickly discovered if the tests had been written before the actual code of the application. The \texttt{Storage}-class was build by creating the tests simultaneously to the code, which most probably increased productivity and decreased the amount of bugs in the class.

\chapter{Result}
\label{sec:result}

\section{View, Controller and Startup}
\label{sec:sup}

In order to launch the application, there is need for a class handling the initialization of the controller and the view (which isn't fully covered in this report) and containing a \texttt{public static void main(String[] args)} function to be able to run the application from command-line. The class containing main is available in the \texttt{CacheSimulator}-class which source-code is available in subsection \ref{subsec:cachesimulator.java}.

The \texttt{CacheSimulator}-class initializes an object of the class \texttt{Controller} (which source is in subsection \ref{subsec:controller.java}) which handles the communication between the model classes and the \texttt{View}.


\subsection{Class: CacheSimulator}
\label{subsec:cachesimulator.java}

The \texttt{CacheSimulator} bootstraps the \texttt{Controller} (in \ref{subsec:controller.java}) and the \texttt{View} (in \ref{subsec:view.java})

\lstinputlisting[language=Java]{cachesim/source/is/mjuk/cache/CacheSimulator.java}

\subsection{Class: Controller}
\label{subsec:controller.java}

The \texttt{Controller} is responsible for the connection between the different models and any view.

\lstinputlisting[language=Java]{cachesim/source/is/mjuk/cache/Controller.java}

\subsection{Class: View}
\label{subsec:view.java}

The \texttt{View} simply keeps the application \texttt{Controller} (in \ref{subsec:controller.java}) connected to the user by presenting user friendly alternatives.

\lstinputlisting[language=Java]{cachesim/source/is/mjuk/cache/View.java}

\subsection{Class: Storage}
\label{subsec:storage.java}

The \texttt{Storage} is a non-persistent, singleton\footnote{i.e. there may ever only be a single \texttt{Storage}-object in existance in any one application.} storage unit, which keeps track of every single cache instruction (storing \texttt{InstructionDTO}s [which are referred to in \ref{subsec:instructiondto.java}]) and a couple of other data related to the application. The \texttt{Storage}-singleton may produce the vast majority of the content in a \texttt{SimulationDTO} (in \ref{subsec:simulationdto.java}) summarizing a single execution of the application. The \texttt{Storage}-object is written using TDD\footnote{The tests were written before the code} and it's tests are available at \ref{subsec:storagetest.java})

\lstinputlisting[language=Java]{cachesim/source/is/mjuk/cache/Storage.java}

\subsection{Class: SimulationDTO}
\label{subsec:simulationdto.java}

The \texttt{SimulationDTO} is a data transfer object that keeps track of a summarized essential data of a single run of a cache simulation.

\lstinputlisting[language=Java]{cachesim/source/is/mjuk/cache/SimulationDTO.java}

\section{User}
\label{sec:user}

The User-handling part of the application is the simplest, and is just taking care of a single string-object, the \textit{nickname} and a \textit{datetime}-stamp.

\subsection{Class: User}
\label{subsec:user.java}
\lstinputlisting[language=Java]{cachesim/source/is/mjuk/cache/User.java}

\section{Cache- and AddressLayout}
\label{sec:cache}

During the second phase of the application's runtime, the \texttt{Controller} (from subsection \ref{subsec:controller.java}) creates a \texttt{DataCache} (in \ref{subsec:datacache.java}) based on a user specified \texttt{CacheLayout} (in \ref{subsec:cachelayout.java}). It also associates an \texttt{AddressLayout} (in subsection \ref{subsec:addresslayout.java}) which can be used on the virtual cache contained within the \texttt{DataCache}.

The controller also stores the \texttt{CacheLayout} and \texttt{AddressLayout} data as a \texttt{LayoutDTO} (in \ref{subsec:layoutdto.java}) in the \texttt{Storage} singleton (from \ref{subsec:storage.java}) 

\subsection{Class: CacheLayout}
\label{subsec:cachelayout.java}

The \texttt{CacheLayout} is created by the controller with a \textit{BlockSize}, \textit{BlockCount} and \textit{Associativity}. The block size and the block count has to be powers of two while the associativity may be anything to your liking.

The \texttt{CacheLayout} creates a \texttt{AddressLayout} (in \ref{subsec:addresslayout.java}) by calculating it's properties from the \textit{BlockSize} and \textit{BlockCount} properties of the \texttt{CacheLayout}.
The following formulae is used to determine the size of the index and offset sizes of the \texttt{AddressLayout}

\begin{algorithmic}
\State $x \in \mathbb{R} \geq 0$ 
\If {$i \in 2^x$}
    \State $\text{size of address element} \gets \log _2 i$ 
\EndIf
\end{algorithmic}

Where \textit{i} is the \textit{BlockSize} in case of the offset and the \textit{BlockCount} for the index. In the case that either the block size or the block count isn't a power of two, the operation will fail with an IllegalArgumentException.

The final \texttt{AddressLayout} property is the tag size, which is $32 - (\text{indexSize} + \text{offsetSize})$ (assuming the 32-bit processor the simulator emulates).

The \texttt{CacheLayout} is tested in \ref{subsec:cachelayouttest.java}.

\lstinputlisting[language=Java]{cachesim/source/is/mjuk/cache/CacheLayout.java}

\subsection{Class: AddressLayout}
\label{subsec:addresslayout.java}

The \texttt{AddressLayout} contains three objects, the \textit{tag size}, the \textit{index size} and the \textit{offset size}. These values are calculated from the \textit{BlockSize} and \textit{BlockCount} from the \texttt{CacheLayout} (in \ref{subsec:cachelayout.java}).

\lstinputlisting[language=Java]{cachesim/source/is/mjuk/cache/AddressLayout.java}

\subsection{Class: LayoutDTO}
\label{subsec:layoutdto.java}

The \texttt{LayoutDTO} is a data transfer object that contains data regarding the properties of \texttt{AddressLayout} (in \ref{subsec:addresslayout.java}) and the \texttt{CacheLayout} (in \ref{subsec:cachelayout.java}). It stores the block size, block count, associativity, tag size, index size and offset size of a single cache.

\lstinputlisting[language=Java]{cachesim/source/is/mjuk/cache/LayoutDTO.java}

\subsection{Class: AddressDTO}
\label{subsec:addressdto.java}

The \texttt{AddressDTO} is for transferring a parseable address. It is created by splitting an address to it's respective block size, block count and associativity fields.

\lstinputlisting[language=Java]{cachesim/source/is/mjuk/cache/AddressDTO.java}

\section{Instructions}

\subsection{Enum: InstructionType}
\label{subsec:instructiontype.java}

The \texttt{InstructionType}-type contains either \textit{STORE} or \textit{LOAD} and denotes if a single instruction was/is a store or load instruction.

\lstinputlisting[language=Java]{cachesim/source/is/mjuk/cache/InstructionType.java}

\subsection{Class: Instruction}
\label{subsec:instruction.java}

The \texttt{Instruction}-class executes a single instruction to a \texttt{DataCache} (in \ref{subsec:datacache.java}). When execution is done, the execution data is stored in an \texttt{InstructionDTO} (in \ref{subsec:instructiondto.java}) and sent to the callee.

\lstinputlisting[language=Java]{cachesim/source/is/mjuk/cache/Instruction.java}

\subsection{Class: InstructionDTO}
\label{subsec:instructiondto.java}

The \texttt{InstructionDTO} contains data about a single instruction with boolean information about if the instruction was a hit or a miss, and data about which address it was specified toward in the cache and if it were a load or store instruction.

\lstinputlisting[language=Java]{cachesim/source/is/mjuk/cache/InstructionDTO.java}

\subsection{Class: DataCache}
\label{subsec:datacache.java}

The \texttt{DataCache} is a container for a number of \texttt{Block}s (in \ref{subsec:block.java}). The object also keeps track on the amount of hits, misses, stores and loads during it's lifetime.

The \texttt{DataCache} is tested in \ref{subsec:datacachetest.java}.

\lstinputlisting[language=Java]{cachesim/source/is/mjuk/cache/DataCache.java}

\subsection{Class: Block}
\label{subsec:block.java}

A \texttt{Block}-objects is a lightweight single-position object within a cache. The objects are created by the \texttt{DataCache} (in \ref{subsec:datacache.java}) which calculates the correct amount of blocks from the \texttt{CacheLayout} (in \ref{subsec:cachelayout.java}) properties BlockCount and Associativity. There are $\text{BlockCount} * \text{Associativity}$ \texttt{Block}s in a single \texttt{DataCache}.

\lstinputlisting[language=Java]{cachesim/source/is/mjuk/cache/Block.java}

\section{Tests}

\subsection{Class: CacheLayout Test}
\label{subsec:cachelayouttest.java}

Test for the \texttt{CacheLayout}-class in \ref{subsec:cachelayout.java}.

\lstinputlisting[language=Java]{cachesim/test/is/mjuk/cache/CacheLayoutTest.java}

\subsection{Class: DataCache Test}
\label{subsec:datacachetest.java}

Test for the \texttt{DataCache}-class in \ref{subsec:datacachetest.java}.

\lstinputlisting[language=Java]{cachesim/test/is/mjuk/cache/DataCacheTest.java}

\subsection{Class: Storage Test}
\label{subsec:storagetest.java}

Test for the \texttt{Storage}-class in \ref{subsec:storagetest.java}.

\lstinputlisting[language=Java]{cachesim/test/is/mjuk/cache/StorageTest.java}

\section{Output from a run}
\lstinputlisting[language={}]{run.txt}

\section{Utilities}

\subsection{Class: MisMath}
\label{subsec:mismath.java}
\lstinputlisting[language=Java]{cachesim/source/is/mjuk/utils/MisMath.java}

\end{document}
